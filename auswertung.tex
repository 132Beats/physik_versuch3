\newpage
\section{ Auswertung der Messergebnisse und Fehlerrechnung}

\subsection{Galton-Brett}
\subsubsection{Vergleich des Gesamtergebnisses mit der Erwartung und dem ersten Versuch}
Beim Galton-Brett wurde als Idealergebnis eine Normalverteilung der Eisenkugeln in den Ausgabefächern erwartet. Jedoch ist bereits bei der Durchführung des Versuches eine deutliche Verschiebung der Werte auf die linke Seite aufgefallen:

\begin{figure}[htb]
    \centering
    \includesvg[width=\textwidth]{diagrams/500_vs_1_vs_ideal.svg}
    \caption[Grafische Darstellung der Verteilung]{Grafische Darstellung des Erwartungswertes, Gesamtergebnis und Ergebnis aus dem ersten Versuch}
\end{figure}

Wie der Grafik zu entnehmen ist, ist die Abweichung zwischen den Ergebnissen des ersten Versuches und der Erwartungskurve sehr groß. Dies liegt, wie in der Versuchsanleitung beschrieben, daran, dass sich eine Binominalverteilung erst für große Wiederholungen bildet:
\begin{quotation}
    \centering
    \glqq Eine solche Verteilung ist allerdings erst dann zu beobachten, wenn viele Kugeln diesen Prozess durchlaufen haben. Eine Einzelmessung wird uns kaum Informationen über diesen Prozess geben.\grqq 
    \newline \textit{- Anleitung zum Physikpraktikum, Abschnitt 1.3}
\end{quotation}

\begin{table}[htb]
    \centering
    \csvloop{
    file=tables/differenz_galton.csv,
    column count=15,
    before reading=\begin{tabular}{|l|llllllllllllll|}\hline,
    /csv/separator = semicolon,
    command = \csvlinetotablerow,
    late after line=\\,
    late after last line=\\\hline,
    after reading=\end{tabular}
    }
    \caption{Darstellung der Differenz zwischen Erwartung und Ergebnis}
\end{table}


