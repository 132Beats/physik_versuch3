\newpage
\section{ Auswertung der Messergebnisse und Fehlerrechnung}

\subsection{Feder-Masse-Pendel}

\subsubsection{Statische Messung}
Gesucht ist die Federkonstante.
\newline
D =: Federkonstante
\newline
m =: Masse
\newline
g =: Erdbeschleunigung
\newline
l =: Auslenkung
\begin{equation}
D = \frac{m\cdot g}{l}
\end{equation}
Da die Formel nur eine mit Fehlern behaftete Größe beinhaltet (die Auslenkung l), entspricht der relative Fehler der Federhärte dem relativen Fehler der Auslenkung.

\begin{table}[thb]
	\centering
	\hline
	\csvloop{
		file=tables/D1.csv,
		no head,
		column count=3,
		before reading=\begin{tabular}{|l|l|l|l|l|l|l|l|l|l|l|l|l|l|l|l|l|},
			/csv/separator = semicolon,
			command = \csvlinetotablerow,
			late after line=\\,
			late after first line=\\\hline,
			late after last line=\\\hline,
			respect percent = true,
			after reading=\end{tabular}
	}
	%\caption{Alle Zeitmessungen in Sekunden}
\end{table}
\begin{table}[thb]
	\centering
	\hline
	\csvloop{
		file=tables/D2.csv,
		no head,
		column count=3,
		before reading=\begin{tabular}{|l|l|l|l|l|l|l|l|l|l|l|l|l|l|l|l|l|},
			/csv/separator = semicolon,
			command = \csvlinetotablerow,
			late after line=\\,
			late after first line=\\\hline,
			late after last line=\\\hline,
			respect percent = true,
			after reading=\end{tabular}
	}
	%\caption{Alle Zeitmessungen in Sekunden}
\end{table}

\subsubsection{Dynamische Messung}
Die Schwingungsgleichung der Feder:
\begin{equation}
T = 2\pi \sqrt{\frac{m}{D}}
\end{equation}
Umgestellt zur Federhärte ergibt sich:
\begin{equation}
D = \frac{4\pi^{2}m}{T^{2}}
\end{equation}
Da die Formel nur eine mit Fehlern behaftete Größe beinhaltet (die Periodendauer T), entspricht der relative Fehler der Federhärte dem dopelten (aufgrund der Potenz) relativen Fehler der Periodendauer.
\begin{table}[thb]
	\centering
	\hline
	\csvloop{
		file=tables/D1d.csv,
		no head,
		column count=3,
		before reading=\begin{tabular}{|l|l|l|l|l|l|l|l|l|l|l|l|l|l|l|l|l|},
			/csv/separator = semicolon,
			command = \csvlinetotablerow,
			late after line=\\,
			late after first line=\\\hline,
			late after last line=\\\hline,
			respect percent = true,
			after reading=\end{tabular}
	}
	%\caption{Alle Zeitmessungen in Sekunden}
\end{table}
\pagebreak
\begin{table}[thb]
	\centering
	\hline
	\csvloop{
		file=tables/D2d.csv,
		no head,
		column count=3,
		before reading=\begin{tabular}{|l|l|l|l|l|l|l|l|l|l|l|l|l|l|l|l|l|},
			/csv/separator = semicolon,
			command = \csvlinetotablerow,
			late after line=\\,
			late after first line=\\\hline,
			late after last line=\\\hline,
			respect percent = true,
			after reading=\end{tabular}
	}
	%\caption{Alle Zeitmessungen in Sekunden}
\end{table}

\subsection{Mathematisches Pendel}
Gesucht ist die Erdbeschleunigung.
\newline
g =: Erdbeschleunigung
\newline
l =: Länge des Pendels
\newline
T =: Periodendauer
\newline
Die Schwingungsgleichung des Pendels:
\begin{equation}
T = 2\pi \sqrt{\frac{l}{g}}
\end{equation}
Umgestellt zur Erdbeschleunigung ergibt sich:
\begin{equation}
g = \frac{4\pi^{2}l}{T^{2}}
\end{equation}
Da die Formel zwei mit Fehlern behaftete Größen beinhaltet (die Periodendauer T und die Pendellänge l), entspricht der relative Fehler der Erdbeschleunigung der Summe der relativen Fehler der Periodendauer und der doppelten (aufgrund der Potenz) Pendellänge.
\begin{table}[thb]
	\centering
	\hline
	\csvloop{
		file=tables/g.csv,
		no head,
		column count=3,
		before reading=\begin{tabular}{|l|l|l|l|l|l|l|l|l|l|l|l|l|l|l|l|l|},
			/csv/separator = semicolon,
			command = \csvlinetotablerow,
			late after line=\\,
			late after first line=\\\hline,
			late after last line=\\\hline,
			respect percent = true,
			after reading=\end{tabular}
	}
	%\caption{Alle Zeitmessungen in Sekunden}
\end{table}
%\subsubsection{Vergleich des Gesamtergebnisses mit der Erwartung und dem ersten Versuch}
%Beim Galton-Brett wurde als Idealergebnis eine Normalverteilung der Eisenkugeln in den Ausgabefächern erwartet. Jedoch ist %bereits bei der Durchführung des Versuches eine deutliche Verschiebung der Werte auf die linke Seite aufgefallen:

%\begin{figure}[htb]
%    \centering
%    \includesvg[width=\textwidth]{diagrams/500_vs_1_vs_ideal.svg}
%    \caption[Grafische Darstellung der Verteilung]{Grafische Darstellung des Erwartungswertes, Gesamtergebnis und Ergebnis aus %dem ersten Versuch}
%\end{figure}

%Wie der Grafik zu entnehmen ist, ist die Abweichung zwischen den Ergebnissen des ersten Versuches und der Erwartungskurve sehr %groß. Dies liegt, wie in der Versuchsanleitung beschrieben, daran, dass sich eine Binominalverteilung erst für große %Wiederholungen bildet:
%\begin{quotation}
 %   \centering
 %   \glqq Eine solche Verteilung ist allerdings erst dann zu beobachten, wenn viele Kugeln diesen Prozess durchlaufen haben. %Eine Einzelmessung wird uns kaum Informationen über diesen Prozess geben.\grqqy
%    \newline \textit{- Anleitung zum Physikpraktikum, Abschnitt 1.3}
%\end{quotation}

%\begin{table}[htb]
%    \centering
%    \csvloop{
%    file=tables/differenz_galton.csv,
%    column count=15,
%    before reading=\begin{tabular}{|l|llllllllllllll|}\hline,
%    /csv/separator = semicolon,
%    command = \csvlinetotablerow,
%    late after line=\\,
%    late after last line=\\\hline,
%    after reading=\end{tabular}
%    }
%    \caption{Darstellung der Differenz zwischen Erwartung und Ergebnis}
%\end{table}


