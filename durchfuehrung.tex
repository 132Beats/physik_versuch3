\section{Durchführung der Messung}

\subsection{Feder-Masse-Pendel}
Während des Versuchs wurden die Auslenkungen verschiedener Massen notiert sowie 5 Messungen der zeit für eine bestimmte Anzahl an Schwingungen.
\newline
m =: Angehängte Masse
\newline
l =: Auslenkung
\newline
n =: Anzahl der Perioden

\subsubsection{Feder Nr. 1}
\begin{table}[thb]
  \centering
  \hline
  \csvloop{
    file=tables/Feder1.csv,
    no head,
    column count=11,
    before reading=\begin{tabular}{|l|l|l|l|l|l|l|l|l|l|l|l|l|l|l|l|l|},
    /csv/separator = semicolon,
    command = \csvlinetotablerow,
    late after line=\\,
    late after first line=\\\hline,
    late after last line=\\\hline,
    respect percent = true,
    after reading=\end{tabular}
  }
  \caption{Alle Zeitmessungen in Sekunden}
\end{table}
\subsubsection{Feder Nr. 2}
\begin{table}[thb]
	\centering
	\hline
	\csvloop{
		file=tables/Feder2.csv,
		no head,
		column count=11,
		before reading=\begin{tabular}{|l|l|l|l|l|l|l|l|l|l|l|l|l|l|l|l|l|},
			/csv/separator = semicolon,
			command = \csvlinetotablerow,
			late after line=\\,
			late after first line=\\\hline,
			late after last line=\\\hline,
			respect percent = true,
			after reading=\end{tabular}
	}
\caption{Alle Zeitmessungen in Sekunden}
\end{table}

\subsection{Mathematisches Pendel}
In diesem Experiment wurden an 5 verschieden langen Pendel 10 Periodendauern gemessen.
\newline
l =: Länge des Pendels
\newline
n =: Anzahl der Perioden
\begin{table}[thb]
	\centering
	\hline
	\csvloop{
		file=tables/mpendel.csv,
		no head,
		column count=7,
		before reading=\begin{tabular}{|l|l|l|l|l|l|l|l|l|l|l|l|l|l|l|l|l|},
			/csv/separator = semicolon,
			command = \csvlinetotablerow,
			late after line=\\,
			late after first line=\\\hline,
			late after last line=\\\hline,
			respect percent = true,
			after reading=\end{tabular}
	}
\caption{Alle Zeitmessungen in Sekunden}
\end{table}

%\csvautolongtable[
%  /csv/separator = semicolon,
%]{tables/messwerte_freier_fall.csv}