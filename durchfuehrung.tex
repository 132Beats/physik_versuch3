\section{Durchführung der Messung}
 Geben Sie hier die Messresultate in tabellarischer Form an.

\subsection{Galton-Brett}
Für den Versuch am Galton-Brett wurden 50 Eisenkugeln nacheinander am Einlass eingeführt und nach erfolgreichem Durchlauf aller Kugeln die Anzahl in jedem Ausgabefach gezählt. Dieser Wert wurde neben dem Namen des Durchführenden dokumentiert.
Dieser Vorgang wurde 10 mal im Wechsel des Durchführenden durchgeführt.
\newline
In diesem Versuch wurde das Brett Nr. 4 verwendet.

\subsubsection{Ermittelte Werte}
Nach erfolgreicher Durchführung der 10 Versuche ergeben sich die folgenden Werte:
\begin{table}[thb]
  \centering
  \hline
  \csvloop{
    file=tables/messwerte_galton.csv,
    no head,
    column count=17,
    before reading=\begin{tabular}{|l|l|l|l|l|l|l|l|l|l|l|l|l|l|l|l|l|},
    /csv/separator = semicolon,
    command = \csvlinetotablerow,
    late after line=\\,
    late after first line=\\\hline,
    late after last line=\\\hline,
    respect percent = true,
    after reading=\end{tabular}
  }
  \caption{Messwerte aller Versuche am Galton-Brett}
\end{table}

\subsection{Freier Fall}
Beim freien Fall wurde eine Eisenkugel 50 mal zwischen zwei Lichtschranken fallen gelassen. Der Abstand der Lichtschranken betrug 856 +- 1mm.
Die Abstandsmessung wurde alle 10 Messungen wiederholt.

\subsubsection{Gemessene Werte}
Auf 50 Wiederholungen der Messung wurden folgende Werte gemessen:

\csvautolongtable[
  /csv/separator = semicolon,
]{tables/messwerte_freier_fall.csv}